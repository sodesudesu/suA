\documentclass[12pt]{ltjsarticle}

\usepackage[hiragino-pro]{luatexja-preset}
%\usepackage{lmodern} % 数学ガールの真似
%\usepackage{ccfonts} % ccfonts を入れると、sin, cosだけじゃなくすべての英文がConcreteになってしまうので注意
%\usepackage[euler-digits]{eulervm}
\usepackage{amsmath}

\begin{document}
\pagestyle{empty}


%\usepackage[T1]{fontenc}
%\usepackage{textcomp}
% \usepackage[utf8]{inputenc}
% \usepackage{amsmath,amssymb}
% \pagestyle{empty}

{\LARGE%
\noindent
\textbf{%
「余事象の確率」補足資料
}}

\begin{flushleft}
2024年6月3日 \\
\end{flushleft}

\section*{ド・メレの問題}

確率論は、賭博師のシュヴァリエ・ド・メレ\footnote{%
  Chevalier de Méré  (1607--1684)。本名は Antoine Gombaud。
  フランスの貴族・賭博師。
}
がおこなっていた賭けに関するパスカル\footnote{%
  Blaise Pascal (1623--1662)。
  フランスの思想家・科学者。
  思想家としては、「人間は考える葦である」という言葉で有名だろう。
  数学者としてはパスカルの三角形や円錐曲線論の発表、物理学者としては流体力学のパスカルの原理の発見など、多くの分野で業績を残した。
}
とフェルマー\footnote{%
  Pierre de Fermat (1601--1665)。
  フランスの弁護士・数学者。
  本業は弁護士で数学は趣味として嗜んでいたという。
  彼が本の余白に書き残したフェルマーの最終定理(3以上の自然数 $n$ に対して $x^n + y^n = z^n$ を満たす自然数の組 $(x, y, z)$ は存在しない)は、彼の死後300年以上経った1995年にやっと証明が与えられた。
  フェルマーの最終定理について興味があれば、サイモン・シン『フェルマーの最終定理』(2006, 新潮文庫)を読んでほしい。
}
の手紙のやりとりから発展したと言われている。
ここでは、彼らの議題の1つであったサイコロの問題を紹介する\footnote{%
  もう1つのテーマは、サイコロのゲームを中断したとき、賞金をどう配分するのが公平かという問題だったという。
}
。

シュヴァリエ・ド・メレは次の2つの賭けをおこなっていた:
\begin{enumerate}
\item サイコロを4回投げて、1回でも6の目が出たら勝ち。
\item サイコロ2つを同時に24回投げて、1回でもゾロ目 (6, 6) が出たら勝ち。
\end{enumerate}
まだ確率論が整理される前であったが、ド・メレは経験的にこの2つの賭けの勝率が異なっていると勘づいていたらしい。
この2つのゲームの勝率をそれぞれ計算し、どちらの方が勝ちやすいか調べよ。
必要に応じて計算機の助けを借りながら取り組んでほしい。

\end{document}