\documentclass[12pt]{ltjsarticle}

\usepackage[hiragino-pro]{luatexja-preset}
%\usepackage{lmodern} % 数学ガールの真似
%\usepackage{ccfonts} % ccfonts を入れると、sin, cosだけじゃなくすべての英文がConcreteになってしまうので注意
%\usepackage[euler-digits]{eulervm}
\usepackage{amsmath,amssymb}
\usepackage[margin=20truemm]{geometry}

\begin{document}


{\LARGE%
\noindent
\textbf{%
領域と最大・最小
}}

\begin{flushleft}
2024年6月4日 \\
\end{flushleft}

\noindent
\textbf{重要例題122}~
$x,\, y$ が 3つの不等式 $3x-5y \geqq -16,\, 3x-y \leqq 4,\, x+y \geqq 0$ を満たすとき、$2x+5y$ の最大値と最小値を求めよ。


\newpage

\noindent
\textbf{練習122 (2)}~
$x,\, y$ が 連立不等式 $x+y \geqq 1,\, 2x+y \leqq 6,\, x+2y \leqq 4$ を満たすとき、$2x+3y$ の最大値と最小値を求めよ。

\newpage

\noindent
\textbf{重要例題124}~
$x,\, y$ が 2つの不等式 $x^2 + y^2 \leqq 10,\, y \geqq -2x + 5$ を満たすとき、$x+y$ の最大値と最小値を求めよ。

\newpage

\noindent
\textbf{練習124}~
座標平面上で不等式 $x^2 + y^2 \leqq 2,\, x+y \geqq 0$ で表される領域を $A$ とする。
点 $(x,\,y)$ が $A$ 上を動くとき、$4x+3y$ の最大値と最小値を求めよ。

\end{document}