\documentclass[12pt]{ltjsarticle}
\usepackage{lmodern} % 数学ガールの真似
\usepackage{ccfonts} % ccfonts を入れると、sin, cosだけじゃなくすべての英文がConcreteになってしまうので注意
%\usepackage{luatexja-fontspec}
\usepackage[hiragino-pro]{luatexja-preset}
\usepackage{amsmath}
\begin{document}
\pagestyle{empty}

{\Large%
\noindent
\textbf{%
独立な試行の確率と場合分け問題の融合
}}

\begin{flushleft}
$2024$年$6$月$7$日 \\
\end{flushleft}

部室にいた $\mathrm{S}$ 君は、隣の部室にお菓子をもらいに行った。
そこには $\mathrm{A}$ 先輩と $\mathrm{B}$ 先輩がいて、自分たちと腕相撲を $3$ 回して $2$ 連勝した時点でお菓子をあげるという。
$\mathrm{S}$ 君は対戦順序を「$\mathrm{A}$ 先輩 -- $\mathrm{B}$ 先輩 -- $\mathrm{A}$ 先輩」もしくは「$\mathrm{B}$ 先輩 -- $\mathrm{A}$ 先輩 -- $\mathrm{B}$ 先輩」から選べる。
$\mathrm{S}$ 君が $\mathrm{A}$ 先輩に勝つ確率は $\dfrac{1}{2}$ で、$\mathrm{B}$ 先輩に勝つ確率は $\dfrac{1}{3}$ である。

$\mathrm{S}$ 君は、$\mathrm{A}$ 先輩なら勝ちやすいと考えて、$\mathrm{A}$ 先輩とより多く対戦する「$\mathrm{A}$ 先輩 -- $\mathrm{B}$ 先輩 -- $\mathrm{A}$ 先輩」の順が有利だと考えた。
この選択に対する説明として最も適切なのは、次のうちどれ?
(統計検定$2$級公式問題集より)

\begin{enumerate}
\item 「$\mathrm{A}$ 先輩 -- $\mathrm{B}$ 先輩 -- $\mathrm{A}$ 先輩」の順の方がお菓子を獲得する確率は高いので、$\mathrm{S}$ 君の選択は好ましい。
\item 「$\mathrm{B}$ 先輩 -- $\mathrm{A}$ 先輩 -- $\mathrm{B}$ 先輩」の順の方がお菓子を獲得する確率は高いので、$\mathrm{S}$ 君の選択は好ましくない。
\item どちらの選択をしてもお菓子を獲得する確率は変わらないので、$\mathrm{S}$ 君の選択でも問題はない。
\end{enumerate}

\end{document}